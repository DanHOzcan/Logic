\documentclass{article}
\title{Functions, Alphabets and Models}
\usepackage{amssymb,stackengine,amsmath,mathrsfs,amsthm,bm}
\usepackage{setspace} \doublespacing
\begin{document}
\maketitle
\newtheorem{theorem}{Theorem}
\newtheorem{proposition}{Proposition}
\newtheorem{lemma}[theorem]{Lemma}

I will prove five main propositions, building up to key properties of first-order models. Each time I will collect together and prove various lemmas needed for upcoming proposition.

\begin{lemma}
Suppose $ f : a \rightarrow b$. If $f$ is bijective, then $f^{-1}: b \rightarrow a$ and $f^{-1}$ is bijective. \end{lemma}
\begin{proof}
Let $ f : a \rightarrow b$ and suppose $f$ is bijective. By the definition of the inverse $f^{-1}: Ran(f) \rightarrow a$ exists. But in this case where $f$ is bijective, $Ran(f) = b$, and therefore
$f^{-1}: b \rightarrow a$. We now show that $f^{-1}$ is both injective and surjective. To see that it is injective, suppose $f^{-1}(x) = f^{-1}(y)$. Then, by the definition of the inverse(cite), $f(f^{-1}(y)) = x$. But since $f^{-1}(x) = f^{-1}(y)$, we also have $f(f^{-1}(y)) = y$ by substitution. Therefore, by the transitivity of identity, $x = y$. Since $x$ and $y$ were arbitrary, we can conclude that $f^{-1}$ is injective. \\ \\To see that $f^{-1}$ is surjective, suppose that $x \in a$. By assumption there is a function $ f : a \rightarrow b$, and therefore by the existence condition on a function, $\exists y \in b f(x) = y$. Then – again by the definition of the inverse – we know that $f^{-1}(y) = x$. Since $x$ was arbitrary, we can conclude that $f^{-1}$ is surjective. So $f^{-1}$ is bijective, as required. 
\end{proof}For this next lemma, define $i_a: a \rightarrow a$ where $i_a(x) = x$ to be the identity function on $a$. Plainly $i_a$ is bijective.
\begin{lemma}
Let $f : a \rightarrow b$ and suppose $f$ is bijective. Then $f^{-1} \circ f = i_a$ and $f^{-1} \circ f$  is bijective \end{lemma}
\begin{proof}
Let $f : a \rightarrow b$ be arbitrary and suppose $f$ is bijective. Then from our proof of Lemma 1, we know there there exists $f^{-1}: b \rightarrow a$ and that $f^{-1}$ is bijective. Now, since $Dom(f^{-1}) = Ran(f)$, by the definition of composition (Unit 2: 15), there exists $f^{-1} \circ f : a \rightarrow a$ such that $f^{-1} \circ f(x) = f^{-1}(f(x))$. By Lemma 2, $f^{-1} \circ f$ is bijective.
\end{proof}
\begin{lemma}
Let $f: a \rightarrow b$ and $g: b \rightarrow a$. If $\forall x \in a \hspace{0.1cm} g \circ f(x) = x$, then $f$ is injective. \end{lemma}
\begin{proof}
Let $f: a \rightarrow b$ and $g: b \rightarrow a$, and suppose that $\forall x \in a g \circ f(x) = x$; we show that $f$ is injective. To that end, suppose $f(x) = f(y)$. Then $g \circ f(x) = g \circ f(y)$ by simple substitution. But we've assumed that $g \circ f(x) = x$ and $g \circ f(y) = y$, so $x = y$ by the transitivity of identity.
\end{proof}

\begin{proposition} The following definitions of enumerability are equivalent.
\begin{itemize}
\item A set $s$ is enumerable iff either $s = \varnothing$ or there is a surjection $f : \mathbb{Z^+} \rightarrow s$.
\item A set s is enumerable iff there is an injection $g: s \rightarrow \mathbb{Z^+}$.
\end{itemize}
\end{proposition}
\begin{proof}
We show that the two definitions are equivalent by showing that there is an injection $g: s \rightarrow \mathbb{Z^+}$ if and only if either $s = \varnothing$ or there is a surjection $f : \mathbb{Z^+} \rightarrow s$.\\ ($\leftarrow$) We reason by cases. So, suppose $s = \varnothing$. Then there is an injection $g: \varnothing \rightarrow \mathbb{Z^+}$ because $Dom(g) = \varnothing$. Now, suppose instead that there is a surjection $f: \mathbb{Z^+} \rightarrow s$. Then for each $x \in s$, there exists some $y \in \mathbb{Z^+}$ such that $f(y) = x$, and therefore for each $x \in s$, the set $\{ y \in \mathbb{Z^+} : f(y) = x \}$\footnote[1]{To be explicit, we apply separation to $\mathbb{Z^+}$ for each such set} is non-empty. Since each such set is a subset of $\mathbb{N}$, it follows by well-ordering\footnote[2]{note on well-ordering} that all of them have a smallest element. In which case, we can define $g: s \rightarrow \mathbb{Z^+}$ whereby $g(x)$ is the smallest $y \in \mathbb{Z^+}$ such that $g(y) = x$. We can quickly see that this function is bijective, for let $x \in s$.  Then $g(x) =$ the smallest $y \in \mathbb{Z^+}$ such that $f(y) = x$, so $f(g(x)) = f \circ g(x) = x$, and therefore, by Lemma 3,  $g$ is  injective. \\ ($\rightarrow$) This direction is slightly more involved, and we proceed in two steps: first we show that if $g: s \rightarrow \mathbb{Z^+}$ is injective, then $s$ is either finite or there is a bijection $i: \mathbb{Z^+} \rightarrow s$. Second, we show that if either $s$ is finite or $i: \mathbb{Z^+} \rightarrow s$, then either $s = \varnothing$ or there is a surjection $f:  \mathbb{Z^+} \rightarrow s$. We then conclude by the transitivity of conditional implication (add foonote) that if $g: s \rightarrow \mathbb{Z^+}$ is injective, then  $s = \varnothing$ or there is a surjection $f:  \mathbb{Z^+} \rightarrow s$.\\Step 1: \hspace{0.1cm} $g: s \rightarrow  \mathbb{Z^+}$ and suppose $g$ is injective. Let $b = Ran(g) \subseteq  \mathbb{Z^+}$. Then  $h: s \rightarrow b$ is bijective, and, by the definition of equinumerosity, $s \approx b$. Now suppose $b$ is not finite. Then since $b \subseteq N$, we can use well-ordering to define $k:  \mathbb{Z^+} \rightarrow b$ such that \begin{center} $k(n) =$ the smallest number in $b - \{ k(m) \mid m \in \mathbb{Z^+} \wedge m < n \}$. \end{center} This function is bijective; we establish injectivity and surjectivity in turn\\ To see that $h$ is injective, we will prove the contrapositive. So, suppose $i,j \in \mathbb{Z^+}$ and $ i \neq j$. Then, by the definitions of $<$ and $>$ I gave in the first problem sheet, either $i < j$ or $j < i$. Suppose $i < j$. Then, by the definition of $k$, $k(j) \in b - \{k(m) \mid m \in \mathbb{Z^+} \wedge m < j\}$ where $k(i) \in \{k(m) \mid m \in \mathbb{Z^+} \wedge m < j\}$, so $k(i) \neq k(j)$. Now suppose instead that $j < i$. We reason identically: by the definition of $k$, $k(i) \in b - \{k(m) \mid m \in \mathbb{Z^+} \wedge m < i\}$ where $k(j) \in \{k(m) \mid m \in \mathbb{Z^+} \wedge m < i\}$, so $k(i) \neq k(j)$. It follows that $k$ is injective by (the law of) contraposition. \\ To see that $k$ is surjective, suppose $x \in b$. Then since $b$ is infinite, for each $n \in \mathbb{Z^+}, b - \{ h(m) \mid m \in \mathbb{Z^+} \wedge m < n \}$ has a smallest member, but $k$ is precisely the function which assigns the smallest member in that set to some $n \in \mathbb{Z^+}$, so there exists some $n \in \mathbb{Z^+}$ such that $k(n) = x$. In which case, $k$ is bijective. So, by the symmetry and transitivity of equinumerosity, there is a bijection $i: \mathbb{Z^+} \rightarrow s$\\ Step 2: \hspace{0.1cm} Suppose either $s$ is finite. There there exists some set $\{m \in N \mid m \leq n \}$ for the number of elements $n$ in $s$. Clearly we can define a bijection $j: \{m \in N \mid m \leq n \}\rightarrow$\footnote[1]{Come back and fill in later} 
\end{proof}

\begin{center} \textbf{Lemmas for Proposition 2} \end{center}

\begin{lemma} If $f : a \rightarrow b$ and $g : b \rightarrow c$ are both injective, then $g \circ f : a \rightarrow c$ is injective. \end{lemma}
\begin{proof} Let $f : a \rightarrow b$ and $g : b \rightarrow c$, and suppose both are injective; we show that $g \circ f : a \rightarrow c$ is injective too.  To that end, let $x$ and $y$ be arbitrary, and suppose $x,y \in a$ and $g \circ f(x) = g \circ f(y)$. Then, by the definition of composition, $ g \circ f(x) = g(f(x))$ and $g \circ f(y) = g(f(y))$. So, by the transitivity of identity, $g(f(x)) = g(f(y))$. Since $g$ is injective, $f(x) = f(y)$, and since $f$ is injective, $x = y$. Therefore $g \circ f : a \rightarrow c$ is injective. As usual, since $x$ and $y$ are arbitrary, it follows that $g \circ f : a \rightarrow c$ is injective if both $f : a \rightarrow b$ and $g : b \rightarrow c$ are. \end{proof}

\begin{lemma} Let $a$ and $b$ be sets. If $a$ and $b$ are both enumerable, then $a \times b$ is enumerable.\end{lemma}
\begin{proof}
Let $a$ and $b$ be sets, and suppose both are enumerable. Then, from our proof in the question one, we know that there are two injections $f: a \rightarrow \mathbb{Z^+}$ and $g: b \rightarrow \mathbb{Z^+}$. Now let $h: a \times b \rightarrow \mathbb{Z^+} \times \mathbb{Z^+}$ be such such that \begin{center} $h(x,y) = \langle f(x), g(y) \rangle$. \end{center} This function is injective too. To see it, suppose $h(x,y) = h(w,z)$, where all involved are arbitrary.  Then, by the definition of $h$, \begin{center} $\langle f(x), g(y) \rangle = \langle f(w), g(z) \rangle$, \end{center} so by the definition of ordered pairs (Unit 2: 6), $x = w$ and $y = z$. Therefore $\langle x,y \rangle = \langle w,z \rangle$. Finally, by the proof of Theorem 3 ("Cantor's Zig Zag", Unit 2: 16 - 17) together with the first part of our question 1, we know that there is some injection $j : \mathbb{Z^+} \times \mathbb{Z^+} \rightarrow \mathbb{Z^+}$. Therefore $j \circ h: a \times b \rightarrow \mathbb{Z^+}$ is injective to by Lemma 4, so $a \times b$ is enumerable, as required. 
\end{proof}

\begin{lemma} The set of Predicate/relation symbols from the universal alphabet is enumerable. \end{lemma}
\begin{proof}
By the definition of the first-order alphabet (Unit 1: 1), the predicate symbols of the universal alphabet are symbol of the form $P_n^m$, where $m \in \mathbb{Z^+}$ and $n \in \mathbb{N}$. Let $\{ P_n^m \mid m \in \mathbb{Z^+} \wedge n \in \mathbb{N} \}$ be the set of all such symbols, and let \begin{center} $f: \{ P_n^m \mid m \in \mathbb{Z^+} \wedge n \in \mathbb{N} \} \rightarrow \mathbb{Z^+} \times \mathbb{N}$\end{center} such that $f(P_n^m) = \langle m, n \rangle$. This function is clearly injective. What's more, both $\mathbb{Z^+}$ and $\mathbb{N}$ are enumerable, and therefore by Lemma 5,  $\mathbb{Z^+} \times \mathbb{N}$ is enumerable, so there exists some injection $g: \mathbb{Z^+} \times \mathbb{N} \rightarrow \mathbb{Z^+}$. Thus,  by the definition of composition we \begin{center} $g \circ f: \{ P_n^m \mid m \in \mathbb{Z^+} \wedge n \in \mathbb{N} \} \rightarrow \mathbb{Z^+}$ \end{center} and Lemma 4 guarantees this function is injective because both $f$ and $g$ are. So the set of Predicate/relation symbols from the universal alphabet is enumerable, as required. 
\end{proof}
\begin{lemma} The set of individual constants and function symbols from the universal alphabet is enumerable \end{lemma}
\begin{proof}
By the definition of the first-order alphabet, the individual constants and function symbols are symbols of the form $f_y^x$, where $x,y \in \mathbb{N}$. Let $\{f_y^x \mid x,y \in \mathbb{N}\}$ be the set of all such symbols, and let $f: \{f_y^x \mid x,y \in \mathbb{N}\} \rightarrow \mathbb{N} \times \mathbb{N}$ be such that $f(f_y^x) = \langle x, y \rangle$. This function is clearly injective. But, by the same reasoning as above, $\mathbb{N} \times \mathbb{N}$ is enumerable because $\mathbb{N}$, so there is some injection $g:  \mathbb{N} \times \mathbb{N} \rightarrow \mathbb{Z^+}$. Therefore, by Lemma 4, $g \circ f:  \{f_y^x \mid x,y \in \mathbb{N}\} \rightarrow \mathbb{Z^+}$ is injective, so the set of individual constants and function symbols from the universal alphabet is enumerable, as required. 
\end{proof}
\begin{lemma} The first-order alphabet is enumerable \end{lemma}
\begin{proof} The first-order alphabet is a set consisting of the logical terms, individual variables and constants, function and predicate symbols, and, finally, the auxiliary symbols. Let $a$ be the universal alphabet. Then: \begin{center} $a = \{ \neg, \rightarrow, \forall, =\} \cup \{x_n \mid n \in \mathbb{N}\} \cup \{f_n^m \mid m,n \in \mathbb{N}\} \cup \{ P_n^m \mid m \in \mathbb{Z^+} \wedge n \in \mathbb{N} \} \cup \{(,)\}$\footnote[4]{I have removed commas from the last set to avoid confusion} \end{center} Each set here taken by themselves is enumerable. The set of logical terms and the set of auxiliary symbols are enumerable because both are finite, and by Proposition $29$ (Unit 2: 18), all finite sets are enumerable. The set of variables can be enumerated by $f : \mathbb{Z^+} \rightarrow \{x_n \mid n \in \mathbb{N}\} $ where $f(x_n) = x_{n - 1}$. Finally, from our proofs above both the set of constant and function symbols and the set of predicate/relation symbols can be enumerated. Therefore, by four applications of Proposition $28$ – which tells us that the union of two enumerable sets is enumerable – $a$ is also enumerable. So the first-order alphabet is enumerable.
 \end{proof} 
\begin{proposition} 
Show that the set of sentences of first-order logic is enumerable. 
\end{proposition}
\begin{proof} A sentence of first-order logic is a well-formed formula with no variables occurring freely, and a well-formed formula is a finite string of symbols taken from the first-order alphabet (Unit 1: 1). Therefore, a sentence of first-order logic is also a finite string of symbols taken from the first-order alphabet, and the set of all such sentences is a subset of the set of all finite strings taken from the first-order alphabet. By Lemma 8, the first-order alphabet is enumerable, so by Example $30$ the set of all finite strings from the first-order alphabet is enumerable. It follows that the set of all first-order sentences is enumerable, for, by Proposition $30$ (Unit 2: 19), every subset of an enumerable set is itself enumerable. So the set of sentences of first-order logic is enumerable because it is a subset of the set of all finite strings taken from the first-order alphabet. 
\end{proof}
\begin{center} \textbf{Lemmas for Proposition 3} \end{center} In this next proof I will switch back and forth between talking about the elements of a set and talking about the subsets of its power set containing those elements. To avoid confusion I will therefore capitalise when talking about sets. \begin{lemma} Let $A$ and $B$ be sets, and suppose there is some injection $f: A \rightarrow B$. Then there is an injection $g: \mathscr{P}(A) \rightarrow \mathscr{P}(B)$ \end{lemma}
\begin{proof} Suppose $f: A \rightarrow B$, where $f$ is injective and all are otherwise arbitrary. Let $g: \mathscr{P}(A) \rightarrow \mathscr{P}(B)$ be such that \begin{center} $g(X) = \{y \in B \mid \exists x \in X(f(x) = y)\}$\footnote[5]{I think technically we shouldn't have quantifiers in the set definition because it is essentially an abbreviation for a universally quantified sentence with the quantifiers out front (cf. Lavinia's notes: Extensionality and Proposition 1 and 2), but adding the quantifier seems clearer when there are a few variables not mentioned on the left side of the 'such that' line appearing on the right side.} \end{center} Clearly both $g(X) \subseteq \mathscr{P}(B)$. What's more, this function is injective. For suppose $g(X) = g(Y)$. Then \begin{center} $\{y \in B \mid \exists x \in X(f(x) = y)\} = \{y_1 \in B \mid \exists x_1 \in X(f(x_1) = y_1)\}.$ \end{center} Now suppose $x \in X$. Then $f(x) \in g(X)$. But since $g(X) = g(Y)$, there is some $y \in Y$ such that $f(x) = f(y)$, by Extensionality. But since $f$ is injective, it follows that $x = y$, so $x \in Y$. Since $x$ was arbitrary, it follows that $X \subseteq Y$. Similar reasoning – but for a swap of symbols – establishes that $Y \subseteq X$, so $X = Y$. That is to say: if $g(X) = g(Y)$, then $X =Y$, so $g: \mathscr{P}(A) \rightarrow \mathscr{P}(B)$ is injective if $f: A \rightarrow B$ is, as required. 
 \end{proof} 
\begin{proposition}
 Show that the set of all sets of pairs of natural numbers, that is, $\mathscr{P}(\mathbb{N} \times \mathbb{N})$, is not enumerable. 
 \end{proposition}
 \begin{proof} Let $f: \mathbb{Z^+} \rightarrow \mathbb{N} \times \mathbb{N}$ be such that $f(n) = \langle n, 0 \rangle$. This function is clearly injective: if $f(n) = f(m)$, then $\langle n, 0 \rangle = \langle m, 0 \rangle$, which, by the definition of ordered pairs, means that $n = m$. By Lemma 9, there is an injection $g: \mathscr{P}(\mathbb{Z^+}) \rightarrow \mathscr{P}(\mathbb{N} \times \mathbb{N})$. Now, if $\mathscr{P}(\mathbb{N} \times \mathbb{N})$ were enumerable, then there would be another injection $h: \mathscr{P}(\mathbb{N} \times \mathbb{N}) \rightarrow \mathbb{Z^+}$, and in that case, by Lemma 4, $h \circ g:  \mathscr{P}(\mathbb{Z^+}) \rightarrow \mathbb{Z^+}$ would be injective. But this contradicts Cantor's Theorem (Unit 2: 20) that $\mathscr{P}(\mathbb{Z^+})$ is not enumerable. Therefore, $\mathscr{P}(\mathbb{N} \times \mathbb{N})$ is not enumerable, as required.
 \end{proof}
\begin{proposition}
If there is an injective function $f: a \rightarrow b$, and $a$ is not enumerable, then $b$ is not enumerable.
\end{proposition}
\begin{proof} Let $f: a \rightarrow b$, and suppose $a$ is not enumerable. Suppose now that $b$ is enumerable. Then there must be an injection $g: b \rightarrow \mathbb{Z^+}$. By Lemma 4, $g \circ f: a \rightarrow \mathbb{Z^+}$ is injective. But there is no such function because, by assumption, $a$ is not enumerable. Therefore $b$ is not enumerable, as required. \end{proof}
\textbf{Lemmas for Proposition 5}
\begin{lemma} Let $f: a \rightarrow b$ and $g: b \rightarrow c$, and suppose both are surjective. Then $g \circ f: a \rightarrow c$ is surjective. \end{lemma}
\begin{proof} Let $f: a \rightarrow b$ and $g: b \rightarrow c$, and suppose both are surjective. Let $x \in c$ be arbitrary. Because $g$ is surjective, $\exists y \in b(g(y) = x)$, and because $f$ is surjective,  $\exists z \in a(f(z) = y)$. By Substitution, $\exists z \in a(g(f(z)) = x)$, and therefore  $\exists z \in a(g \circ(f(z)) = x)$ by the definition of composition. So $g \circ f$ is surjective. \end{proof}
\begin{lemma} Let $f: a \rightarrow b$ and $g: b \rightarrow c$, and suppose both are bijective. Then $g \circ f: a \rightarrow c$ is bijective. \end{lemma}
\begin{proof} $g \circ f: a \rightarrow c$ is bijective if it is both injective and surjective. Since $f$ and $g$ are both bijective, $f$ and $g$ are also both injective, so $g \circ f$ is injective by Lemma 4. And since $f$ and $g$ are both bijective, $f$ and $g$ are also both surjective, so by Lemma $10$, $g \circ f$ is surjective. In which case $g \circ f$ is bijective, as required. 
\end{proof}
\begin{lemma} Suppose $f: a \rightarrow b$ is bijective. Then $\forall x \in a \hspace{0.1cm} f^{-1} \circ f(x) = x$ \end{lemma}
\begin{proof} By Lemma 1, $f^{-1}: b \rightarrow a$. Now suppose $x \in a$. Then $\exists y \in b f(x) = y$. By the definition of the inverse $f^-1(y) = x$, but since $f(x) = y$, \begin{center}
$f^-1(y) = f^{-1}(f(x)) = f^{-1} \circ f(x) = x$ \end{center} as required. \end{proof}
$\mathcal{M'}$ and $\mathcal{M''}$ 
\begin{proposition} 
Let $\mathcal{M}$ a  be model of $\mathscr{L}$. Then $\mathcal{M} \cong \mathcal{M}$ 
\end{proposition}
\begin{proof} 
Let $\mathcal{M}$ be a model of $\mathscr{L}$, and let $h: | \mathcal{M} | \rightarrow | \mathcal{M} |$ be such that $h(x) = x$. This function is clearly bijective. To see that the other conditions hold, suppose first that $c \in V(\mathscr{L})$. Then $c^\mathcal{M} \in | \mathcal{M} |$, so $h(c^\mathcal{M}) = c^\mathcal{M}$. Next, suppose that $f \in V(\mathscr{L})$ and  $x_1, \ldots, x_n \in | \mathcal{M}|$. Then, since $\forall x \in |\mathcal{M}| (h(x) = x$:
 \begin{align*} 
 h(f^\mathcal{M}(x_1, \ldots, x_n)) &= f^\mathcal{M}(x_1, \ldots, x_n)&&\text{Def. $h: | \mathcal{M} | \rightarrow | \mathcal{M} |$}\\
 &= f^\mathcal{M}(h(x_1), \ldots,h(x_n)).&&\text{Def.  $h: | \mathcal{M} | \rightarrow | \mathcal{M} |$}
 \end{align*} Finally, let $P \in V(\mathscr{L})$ and $x_1, \ldots, x_n \in |\mathcal{M}|$, and suppose \begin{center} $\langle x_1,\ldots, x_n \rangle \in P^\mathcal{M}$. \end{center} Then $\langle (h(x_1), \ldots h(x_n) \rangle \in P^\mathcal{M}$ because, by the definition of $h$ (and the definition of ordered pairs), $\langle (h(x_1), \ldots h(x_n) \rangle  = \langle x_1,\ldots, x_n \rangle$. Conversely, suppose $\langle (h(x_1), \ldots h(x_n) \rangle \in P^\mathcal{M}$. Then, by the same reasoning, $\langle x_1,\ldots, x_n \rangle \in P^\mathcal{M}$ because $\langle (h(x_1), \ldots h(x_n) \rangle =  \langle x_1,\ldots, x_n \rangle$. So $\mathcal{M} \cong  \mathcal{M}$, as required.
\end{proof} \clearpage 
\begin{proposition}
 Let $\mathcal{M}$ and $\mathcal{M'}$  be models of $\mathscr{L}$. Show that if $\mathcal{M} \cong \mathcal{M'}$, then $\mathcal{M'} \cong \mathcal{M}$. 
 \end{proposition}
\begin{proof}  Let $\mathcal{M}$ and $\mathcal{M'}$  be models of $\mathscr{L}$, and suppose $\mathcal{M} \cong \mathcal{M'}$. Then $h: | \mathcal{M} | \rightarrow | \mathcal{M'} |$
is a bijection and the further three conditions on isomorphism hold. By Lemma 1, $h^{-1}: |\mathcal{M'}| \rightarrow |\mathcal{M}|$ is bijective such that $\forall x\in |\mathcal{M}|(h^{-1} \circ h(x) = x)$. To see that the other conditions hold, suppose first that $c \in V(\mathscr{L})$. Then $c^\mathcal{M} \in |\mathcal{M}|$ and $h(c^\mathcal{M}) = c^\mathcal{M'}$, so
\begin{align*}
h^{-1}(c^\mathcal{M'}) &= h^{-1}(h(c^\mathcal{M}))&&\text{Def. of $h: | \mathcal{M} | \rightarrow | \mathcal{M'} |$}\\
&=h^{-1} \circ h(c^\mathcal{M})&&\text{Def. Composition}\\
&= c^\mathcal{M}.&&\text{Lemma 12}
\end{align*}
Next, let $f \in V(\mathscr{L})$ and $x_1', \ldots x_n' \in |\mathcal{M'}|$. Now, since $\mathcal{M} \cong \mathcal{M'}$, we know that \begin{center} $h(f^\mathcal{M}(x_1,\ldots,x_n)) = f^\mathcal{M'}(h(x_1),\ldots,h(x_n))$ \end{center} In which case \begin{center}$h_1(f^\mathcal{M'}(x_1',\ldots,x_n')) = f^\mathcal{M}(x_1,\ldots x_n) = f^\mathcal{M}(h^{-1}(h(x_1)), \ldots h^{-1}(h(x_n)))$, \end{center} as required. Finally, let $P \in V(\mathscr{L})$ and  $x_1', \ldots x_n' \in |\mathcal{M'}|$ Again, since $\mathcal{M} \cong \mathcal{M'}$, we know that \begin{center} $\langle x_1, \ldots, x_n \rangle \in P^\mathcal{M}$ iff $\langle h(x_1), \ldots, h(x_n) \rangle \in P^\mathcal{M'}$ \end{center} But $\langle h(x_1), \ldots h(x_n) \rangle  = \langle x_1', \ldots, x_n' \rangle$ and $\langle x_1, \ldots, x_n \rangle = \langle h^{-1}(x_1'), \ldots, h^{-1}(x_n') \rangle$, so the above is logically equivalent to our goal \begin{center} $\langle x_1', \ldots, x_n\ \rangle \in P^\mathcal{M'}$ iff  $\langle h^{-1}(x_1'), \ldots, h^{-1}(x_n' \rangle \in P^\mathcal{M}$. \end{center} So $\mathcal{M'} \cong \mathcal{M}$, as required.
\end{proof}
5(c) Let $\mathcal{M}$,$\mathcal{M'}$ and $\mathcal{M''}$ be models of $\mathscr{L}$. Show that if $\mathcal{M} \cong \mathcal{M'}$ and $\mathcal{M'} \cong \mathcal{M''}$then $\mathcal{M} \cong \mathcal{M''}$. 
\begin{proof} 
Let $\mathcal{M}$,$\mathcal{M'}$ and $\mathcal{M''}$ be models of $\mathscr{L}$, and suppose that $\mathcal{M} \cong \mathcal{M'}$ and $\mathcal{M'} \cong \mathcal{M''}$. So, there are two bijections $h: | \mathcal{M} | \rightarrow | \mathcal{M'} |$ and $g:  | \mathcal{M'} | \rightarrow | \mathcal{M''}|$ such that the further three conditions hold in both cases. By Lemma $11$, there is a bijection  $g \circ h:  | \mathcal{M}| \rightarrow  | \mathcal{M''}|$ We now show that the further three conditions hold as well. For the first, suppose that $c \in V(\mathscr{L})$ Then $c^\mathcal{M} \in |\mathcal{M}|$ and \begin{center}  $g\circ h(c^\mathcal{M}) = g(h(c^\mathcal{M})) = g(c^\mathcal{M'}) = c^\mathcal{M''}$ \end{center} because 
$\mathcal{M} \cong \mathcal{M'}$ and $\mathcal{M'} \cong \mathcal{M''}$. Next, let $f \in V(\mathscr{L})$ and $x_1, \ldots, x_n \in |\mathcal{M}|$. Then, since $\mathcal{M} \cong \mathcal{M'}$ and $\mathcal{M'} \cong \mathcal{M''}$: 
\begin{align*}
g \circ h(f^\mathcal{M}(x_1, \ldots, x_n)) &= g(h(f^\mathcal{M}(x_1, \ldots, x_n))&&\text{Definition of Composition}\\
&= f^\mathcal{M''}(g(h(x_1)), \ldots,g(h(x_n)))&&\text{by $\mathcal{M} \cong \mathcal{M'}$ and $\mathcal{M'} \cong \mathcal{M''}$}\\
&=  f^\mathcal{M''}(g\circ h(x_1), \ldots,g \circ h(x_n)).&&\text{Definition of Composition}
\end{align*}
Finally, let $P \in V(\mathscr(L))$ and $x_1, \ldots, x_n \in |\mathcal{M}|$, and suppose $\langle x_1, \ldots, x_n \rangle \in P^\mathcal{M}$. Then: 
\begin{align*}
\langle h(x_1), \ldots, h(x_n) \rangle \in &P^\mathcal{M'}&&\text{by $\mathcal{M} \cong \mathcal{M'}$}\\
\langle g(h(x_1)), \ldots g(h(x_n)) \rangle \in &P^\mathcal{M''}&&\text{by  $\mathcal{M'} \cong \mathcal{M''}$}\\
\langle g \circ h(x_1), \ldots, g \circ h(x_n) \rangle \in &P^\mathcal{M''}.&&\text{Definition of composition}
\end{align*}
Conversely, suppose $\langle g \circ h(x_1), \ldots, g \circ h(x_n) \rangle \in P^\mathcal{M''}$. Then 
\begin{align*} 
\langle g(h(x_1)), \ldots g(h(x_n)) \rangle \in &P^\mathcal{M''}&&\text{Definition of Composition}\\
\langle h(x_1), \ldots, h(x_n) \rangle \in &P^\mathcal{M'}&&\text{ $\mathcal{M'} \cong \mathcal{M''}$ and Symmetry of Isomorphism}\\
\langle x_1, \ldots, x_n \rangle \in &P^\mathcal{M}.&&\text{By $\mathcal{M} \cong \mathcal{M'}$ and Symmetry of Isomorphism}
\end{align*}
So $\mathcal{M} \cong \mathcal{M''}$,as required. 
\end{proof}
\end{document}
